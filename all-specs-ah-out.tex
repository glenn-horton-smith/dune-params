\documentclass{article}
\usepackage[top=0.5in, bottom=2.75cm,right=2cm, left=2cm]{geometry}
\usepackage[utf8]{inputenc}
\usepackage[T1]{fontenc}
\usepackage{textcomp}
\usepackage{siunitx}
\usepackage{array}
\usepackage{tabularx}
\usepackage{tabulary}
\usepackage[table]{xcolor}
\RequirePackage{booktabs}
\newcolumntype{L}[1]{>{\raggedright\let\newline\\\arraybackslash\hspace{0pt}}m{#1}}
\newcolumntype{C}[1]{>{\centering\let\newline\\\arraybackslash\hspace{0pt}}m{#1}}
\newcolumntype{R}[1]{>{\raggedleft\let\newline\\\arraybackslash\hspace{0pt}}m{#1}}

\usepackage{etoolbox}
\newcounter{rowcntr}[table]
\renewcommand{\therowcntr}{\thetable.\arabic{rowcntr}}
% A new columntype to apply automatic stepping
\newcolumntype{N}{>{\refstepcounter{rowcntr}\therowcntr}c}
% Reset the rowcntr counter at each new tabular
\AtBeginEnvironment{tabular}{\setcounter{rowcntr}{0}}

\definecolor{dunetablecolor}{RGB}{125,174,211} % matches lighter blue color from Diana's scheme 


\begin{document}

\section{The main table}

This table would enumerate every spec across all subsystem ``tabs''.


\begin{table}[htp]
  \caption{Specification for DSS}
  \centering
  \begin{tabular}{ |l|p{0.45\textwidth}|p{0.25\textwidth}| }
    \hline
        

    \rowcolor{dunetablecolor}
ID:  DSS 1   & Name: Gap width between adjacent APAs  & Spec:  $<$\SI{20}{\milli\meter} \\
    \hline
    Rationale &  \multicolumn{2}{p{0.7\textwidth}|}{ Some gap is necessary to allow for detector shrinkage. Installation of electron diverters can mitigate effects of small gaps on reconstruction. } \\
    \hline
ProtoDUNE Validation & \multicolumn{2}{p{0.7\textwidth}|}{ Example: "Simulations of 10-50 mm gaps showed slight and gradual drop in efficiency over 20 mm." } \\
    \hline
Simulation Validation & \multicolumn{2}{p{0.7\textwidth}|}{ Current conceptual design has 17mm gaps between each grouping of three APAs } \\
        \hline
   

    \rowcolor{dunetablecolor}
ID:  DSS 2   & Name: Drift field uniformity throughout volume  & Spec:  $<1$\%  \\
    \hline
    Rationale &  \multicolumn{2}{p{0.7\textwidth}|}{ Inhomogeneities in the field will degrade reconstruction performance. Misalignments of TPC components could introduce drift field non-uniformities beyond those specified in the HVS requirements. } \\
    \hline
ProtoDUNE Validation & \multicolumn{2}{p{0.7\textwidth}|}{ Example: "Simulations of misalignments of $<$1\% do not significantly affect reconstruction." } \\
    \hline
Simulation Validation & \multicolumn{2}{p{0.7\textwidth}|}{ DSS should provide support such that the APA, CPA and FC can maintain rectangular parallelepiped under LAr conditions. Specifically the rails should remain parallel to each other at a fixed distance throughout the detector volume } \\
        \hline
   

    \rowcolor{dunetablecolor}
ID:  DSS 3   & Name: Anne's test  & Spec:  $<19$\%  \\
    \hline
    Rationale &  \multicolumn{2}{p{0.7\textwidth}|}{ Inhomogeneities in the space-time continuum can affect our brains. } \\
    \hline
ProtoDUNE Validation & \multicolumn{2}{p{0.7\textwidth}|}{ Example: "Simulations of brains are wacky, too." } \\
    \hline
Simulation Validation & \multicolumn{2}{p{0.7\textwidth}|}{ DSS should provide support such that the APA, CPA and FC can maintain rectangular parallelepiped under LAr conditions. Specifically the rails should remain parallel to each other at a fixed distance throughout the detector volume } \\
        \hline
   

 
  \end{tabular}
  \label{tab:spectable:DSS}
\end{table}


And here it is referenced as Table~\ref{tab:spectable:DSS}.  
I can still reference a row in the previous table,
Table~\ref{tab:spectable:DSS} (not sure how you'd reference a row other than to say, e.g., ``DSS 2.''

\end{document}